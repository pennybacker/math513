\documentclass[12pt,fleqn,leqno]{exam}

\usepackage{amssymb,amsfonts,amsmath}
\usepackage[letterpaper,margin=1in]{geometry}
\usepackage{graphicx}
\usepackage{tabularx}
\usepackage{enumerate}

\newcommand{\class}{MATH 513/463}
\newcommand{\term}{Spring 2017}
\newcommand{\doctitle}{Homework 9}

\newcommand{\R}{\ensuremath{\mathbb{R}}}
\newcommand{\D}{\ensuremath{\mathrm{R}}}

\def\Xint#1{\mathchoice
	{\XXint\displaystyle\textstyle{#1}}%
	{\XXint\textstyle\scriptstyle{#1}}%
	{\XXint\scriptstyle\scriptscriptstyle{#1}}%
	{\XXint\scriptscriptstyle\scriptscriptstyle{#1}}%
\!\int}
\def\XXint#1#2#3{{\setbox0=\hbox{$#1{#2#3}{\int}$}
	\vcenter{\hbox{$#2#3$}}\kern-.5\wd0}}
\def\ddashint{\Xint=}
\def\dashint{\Xint-}


\parindent 0ex

\pagestyle{head}
\header{\bf \class}{\bf \doctitle\ - Page \thepage\ of \numpages}{\bf \term}
\headrule

\begin{document}

\begin{questions}

\question Solve the following initial-value problems using the method of characteristics.

\begin{parts}
\part $u_x + u_y = u^2$, \quad $u(x, 0) = x$.
\part $(u+y) u_x + u_y = -u$, \quad $u(x, 0) = x$.
\end{parts}

\question Solve the following Cauchy problems using the method of characteristics.

\begin{parts}
\part $x u_x - x u_y = u - 1$, \quad $u(x, x^2) = x^3 + x^2 + 1$.
\part $u u_x + u_y = 2$, \quad $u(x, x) = x$.
\end{parts}

\question Is it possible to solve the problem \[u_x + u_y = 0, \quad u(x, x) = 2\] using characteristics? Solve it or explain why you can't.
\end{questions}

\end{document}
