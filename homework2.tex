\documentclass[12pt,fleqn]{exam}

\usepackage{amssymb,amsfonts,amsmath}
\usepackage[letterpaper,margin=1in]{geometry}
\usepackage{graphicx}
\usepackage{tabularx}

\newcommand{\class}{MATH 513/463}
\newcommand{\term}{Spring 2017}
\newcommand{\doctitle}{Homework 2}

\newcommand{\R}{\ensuremath{\mathbb{R}}}
\newcommand{\D}{\ensuremath{\mathrm{R}}}

\def\Xint#1{\mathchoice
	{\XXint\displaystyle\textstyle{#1}}%
	{\XXint\textstyle\scriptstyle{#1}}%
	{\XXint\scriptstyle\scriptscriptstyle{#1}}%
	{\XXint\scriptscriptstyle\scriptscriptstyle{#1}}%
\!\int}
\def\XXint#1#2#3{{\setbox0=\hbox{$#1{#2#3}{\int}$}
	\vcenter{\hbox{$#2#3$}}\kern-.5\wd0}}
\def\ddashint{\Xint=}
\def\dashint{\Xint-}


\parindent 0ex

\pagestyle{head}
\header{\bf \class}{\bf \doctitle\ - Page \thepage\ of \numpages}{\bf \term}
\headrule

\begin{document}

\begin{questions}

\question In the following way, solve the problem for Poisson's equation on the ball $B(0,1) \subset \R^2$ given by
\[\begin{cases}
\Delta u = x_2 & \text{in } B^0(0,1) \\
u = 1 & \text{on } \partial B(0,1).
\end{cases}\]
Look for a solution in polar coordinates $x_1 = r\cos(\theta)$, $x_2 = r\sin(\theta)$ of the form
\[u(r,\theta) = \frac12\,A_0(r) + \sum_{n = 1}^{\infty} \big( A_n(r)\cos(n\theta) + B_n(r)\sin(n\theta) \big),\]
assuming that $u$ is bounded near the origin.

\begin{parts}
\part Write the function $x_2$ and the operator $\Delta$ in polar coordinates. Deduce the ODEs that $A_0$, $A_1$, $B_1$, \dots~satisfy along with the corresponding initial conditions.
\part Solve these initial value problems and substitute the solutions into the ansatz for $u$. Write your answer as a function of $x_1$ and $x_2$.

\emph{Hint}: You should obtain the ODE $r^2 B_1'' + r B_1' - B_1 = r^3$ when solving for $B_1$. This ODE has a particular solution of the form $B_1(r) = c\,r^3$. Use this fact to obtain the general solution to the ODE.
\end{parts}

\question (Evans \S2.5, \#3) Modify the proof of the mean value formulas to show for $n \geq 3$ that
\[u(0) = \dashint_{\partial B(0,r)} g\,dS + \frac{1}{n(n-2)\alpha(n)} \int_{B(0,r)} \bigg(\frac{1}{|x|^{n-2}} - \frac{1}{r^{n-2}}\bigg)\,f\,dx ,\]
provided
\[\begin{cases}
-\Delta u = f & \text{in } B^0(0,r) \\
u = g & \text{on } \partial B(0,r) .
\end{cases}\]

\question (Evans \S2.5, \#4) We say $v \in C^2(\bar{U})$ is subharmonic if
\[-\Delta v \leq 0 \quad\text{in } U.\]

\begin{parts}
\part Prove for subharmonic $v$ that
\[v(x) \leq \dashint_{B(x,r)} v\,dy \quad\text{for all } B(x,r) \subset U.\]
\part Prove that therefore $\max_{\bar{U}} v = \max_{\partial U} v$.
\part Let $\phi : \R \rightarrow \R$ be smooth and convex. Assume $u$ is harmonic and $v = \phi(u)$. Prove $v$ is subharmonic.
\part Prove $v = |Du|^2$ is subharmonic, whenever $u$ is harmonic.
\end{parts}

\end{questions}

\end{document}
